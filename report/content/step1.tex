%!TEX root = ../report.tex
\section{Skeleton compilation}
    The provided skeleton code has been refactored according to the Model View Controller pattern. This creates a good separation between classes and reduces code clutter. In our specific case, \texttt{model.cpp} and \texttt{model.h} correspond to the model part. \texttt{visualization.cpp} and \texttt{visualization.h} keep track of all the visualization parameters and the visualization of the model data.
    The skeleton code was written in C and has been - partly - rewritten to be compatible C++.

    \subsection*{GLUI}
        In order to show different kind of visualizations, there are some configuration parameters needed. To easily set these parameters, a framework named GLUI \cite{glui} is incorporated in our project. GLUI is able to create check boxes, radio buttons, spinners and more. All these buttons are binded to variables in \texttt{visualization.h}, this creates an organized place where all visualization parameters are set.