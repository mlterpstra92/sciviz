%!TEX root = ../report.tex
\section{Isolines}
Isolines show all points of a scalar dataset's domain where the scalar equals a certain value (or multiple values).
An efficient algorithm to construct these isolines is called the `Marching squares' algorithm, which was applied in our program.
The general idea of the algorithm is to loop over all cells - consisting of four vertices - and check for all vertices if the value at that vertex is above or below a certain threshold, or the isoline value.
After checking the four vertices, 16 different cases can occur, where each case can be coded individually.
The cases determine through which sides fo the grid, if any, a line should be drawn. We then interpolate at each side a line should be drawn to determine the exact position where the grid side has the same value as the isoline threshold.
Since the starting location of a line is the end location of another line from a different cell, a smooth, closed curve is created.