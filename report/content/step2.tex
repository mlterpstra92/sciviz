%!TEX root = ../report.tex
\section*{Step 2}
	\subsection*{New color map}
		Implemented the `bipolar' color map.
		Also, the ability to switch between color maps is added in the UI, using a drop-down menu.
	\subsection*{Color legend}
		The legend for the color map has been implemented.
		On the side of the GLUT subwindow, the reference color map is drawn with the values corresponding to the colors next to it.
	\subsection*{More datasets for color map}
		We provided the means to choose the dataset on which the color mapping is applied to.
		The function \texttt{draw\_smoke} receives a pointer to an array containing the values of the dataset that needs to be visualized.
		There are three different datasets that this pointer can reference:
		\begin{itemize}
			\item Force magnitude
			\item Velocity magnitude
			\item Fluid density
		\end{itemize}
		In case of magnitude visualization, the magnitude of a vector \(a\) is determined by the following equation:
		\[\| a \|\ = \sqrt{a_x^2 + a_y^2}\]
		In case of fluid density, the existing reference to \texttt{rho} in the model can be used.
	\subsection*{Scaling \& clamping}
		Implemented scaling and clamping of color map.
		The user can now choose whether the color map should be scaled to the minimum and maximum of the visualized dataset, or whether the colors should be clamped in a user-specified interval.
	\subsection*{Limit number of colors}
		The ability to limit the number of colors used in the color map has been implemented.
		The color map also only uses these colors.
	\subsection*{HSV color space}
		We also added the feature to change the hue and saturation.
		For this, we used the conversion function from RGB to HSV and vice versa, as described in \cite{telea2014data}.